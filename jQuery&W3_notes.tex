\documentclass[8pt,a4paper]{extarticle}
%\documentclass[8pt,twocolumn,a4paper]{extarticle}
%\documentclass[6pt,a4paper]{article}
\usepackage[utf8]{inputenc}
\usepackage{amsmath}
\usepackage{amsfonts}
\usepackage{amssymb}
\usepackage{graphicx}
\usepackage{enumitem}
\usepackage{setspace} 
\usepackage{float}
\usepackage{url}
\usepackage[usenames,dvipsnames,svgnames,table]{xcolor}
\usepackage{multicol}

\pagecolor{black}
\setlist{nolistsep}
\topmargin -2cm
\oddsidemargin -1.5cm
\evensidemargin -1.5cm
\textwidth 18.75cm
\textheight 26cm
\linespread{1.00}
\color{YellowOrange}

%Other Colors: ForestGreen,Cerulean,Salmon,Purple, Red, White

\twocolumn
\title{Front End Visualization Notes:}
\author{Sean al-Baroudi}

\begin{document}

\newcommand{\mitem}{\item[$\square$]}
\newcommand{\mmitem}{\item[$\triangledown$]}

\maketitle
\section*{aaa}

\begin{itemize}
\item When a webpage is loaded, there is a difference between the \textbf{HTML document}, and the \textbf{DOM model}. The page that you see loaded is an instance of the DOM model, that was interpreted from the HTML document. 
\item Changes in the DOM done dynamically have nothing to do with HTML
\item Hiearchy of Objects in DOM Model:
\begin{enumerate}
\item Window
\item Document
\item HTML
\item BODY...and all its nested objects.

\end{enumerate}
\item
\item
\item
\item
\item
\item
\item
\item
\item
\item

\end{itemize}



\begin{itemize}
\item JQuery is a JavaScript Framework for implementing advanced functionality,
and hiding common boilerplate that occurs from basic JavaScript.
\item JQuery centers around \textit{Selectors}, which are search operations that
find DOM elements and return a set like structure. 
\item Given a set of elements (or single one), \textbf{Actions} can be performed on such elements. Mutation, or filtering of the elements (changing the set being operated on) are common things a developer might do.
\item Selector Syntax:

\begin{verbatim}
 $(selector1,selector2,...).action()
\end{verbatim}

The \textbf{$\$$ }is a symbol that indicates a jQuery call is about to be defined. 

(selector) can be anything - types of DOM tags to unique IDs for elements in our DOM tree. For every selector expression, jQuery searches the DOM tree for matching elements, and appends set results to a chronological list.

(actions) involve callbacks or actual functions, that affect the selected set in some way.

\item You can of course nest and chain jQueries. As actions also return sets, this can be done many times in a row to crush code into one line.


\item Basic JQuery Start up Code:

\begin{verbatim}
$(document).ready(function() {

<put code here>

});

\end{verbatim}

Here, the entire document was selected. an Event Listener has been added, with the condition that it triggers when the DOM model is fully loaded. A anonymous call back function is launched, that executes whatever jQuery commands you specify.


\item \textbf{Attributes vs Properties:} The former refers to meta-data in the HTML document. The latter refers to meta-data in the DOM tree; Properties are mutable; attributes are generally not. Use the \textbf{prop()} function to alter node properties.
\item
\item
\item
\item
\item
\item
\item

\end{itemize}

\subsection*{Understanding Event Handling and Listening (W3 Specification):}

\begin{itemize}
\item The model is very complex. For the dynamic DOM tree, javascript can 
setup event listeners for specific nodes in the tree. \textbf{Every node in the tree can have multiple registered event listeners.} 

\item \textbf{Events per node are registered in order} (so earlier registered events are hit first).
\item You can also have multiple event handlers for the same event type, on the same node; again, they are executed relative to registration order.
\item When a user causes an interaction event on a webpage in the browser, this
causes the following to occur:

\begin{enumerate}
\item A user interaction/browser signal occurs; an event with a possible target node is created.
\item This event is put into the browser's Event Queue.
\item The browser will eventually act on the next element in the Event Queue (pop). The following set of actions occur:
\begin{enumerate}
\item From the DOM root, a Propogation Path to the target node is created (a list of nodes). 
\item \textbf{Capture Phase:} For every ancestor in the Propogation Path, the de-queued event is checked against each ancestor's event listeners. Ancestors who are capturers may  act in this phase.
\item \textbf{At Target Phase:} The actual element that was interacted with is found. The element's listeners are checked with the event. Some may trigger and execute.
\item  \textbf{Bubbling Phase:} Propogation path is backtracked. And each ancestor's event listeners are again checked with the event. They may or may not act again, depending on their configuration.
\end{enumerate}
\item Bubbling Phase can be turned off in jQuery; and registering anscestor event triggers is also optional. For most things, we have event triggers at the actual node the user has interacted with.
\item

\item Low Level Code to add an Event Listener (in Javascript):

\begin{verbatim}
target.addEventListener(EventType, listener, T/F)
\end{verbatim}

jQuery wraps these calls in its library. Target is the element in question, EventType is a user action (click, hover..), listener is our function that responds to the event in a predefined way. True means we have a Capture phase. False means we have Bubbling Phase (the \textit{default} in jQuery).

\end{enumerate}
\end{itemize}

\subsection*{Implementation of W3 model in jQuery:}
\begin{itemize}
\item In jQuery, Bubbling is the default, but Capturing is not.
\item For more fine grained control, there are two stop methods to stop propogation of events.
\begin{enumerate}
\item event.stopPropogation(): This method will allow all registered events at the target to finish executing; but any bubbled ancestor event handlers will be prevented from firing.

\textbf{[Image]}

\item event.stopImmediatePropogation(): Event handlers on the current target will be stopped, in addition to bubbling events with target ancestors.

\textbf{[Image]}

\end{enumerate}
\item
\item
\item
\item
\item
\item
\item
\item
\item
\item
\item
\end{itemize}

\subsection*{Basic Javascript:}
\begin{itemize}
\item Primitive Data Types: String, Boolean, Number.
\item Special Constants: NAN, Undefined.
\item All of the above Primitives \textit{are immutable}; you cannot edit the data "in memory". You make a copy and edit that, and put the transformed data back in place (if you like).
\item All other data types are Objects. There are two views to this: Associative Arrays, and Objects. They are both equivalent, but use different notations.

\item Associative Arrays (AAs): Are Key value pairs that are stored in a list like structure. Notation as follows:

\begin{verbatim}
var myAA = {
	key1:'hello',
	key2:55,
	...
};

OR

var myAA = {};
myAA['hello'] = 12;
myAA['goodbye'] = 13;
\end{verbatim}

The latter method is also used to access the arrays; but we can assign new properties to them as well.

\item Objects are Associative arrays; they really represent syntatic sugar and a different view, when designing data structures. There are three ways to define Objects:
\item Objects store key value pairs like AAs. These can represent \textit{object fields}, or point to functions and represent \textit{object methods}.

\begin{enumerate}
\item Making Singletons: Use the Associative Array method (above)
\begin{verbatim}
var myAA = {
	key1:'hello',
	key2:55,
	...
	
\end{verbatim}

\item Using Object Constructor, adding Properties After:

\begin{verbatim}
var myObj = new Object();
myObj.prop1 = 'hello';
\end{verbatim}

The third method is preferred:

\item Object Constructor Blueprint:

\begin{verbatim}
function myObjConstructor(arg1, arg2, arg3) {
	this.arg1 = arg1;
	....

	this.method1 = function() { ... }
	this.method2 = function() {...}


}

var myObj = new myObjConstructor(1,2....);

\end{verbatim}

\end{enumerate}
\item
\item
\item
\item
\item
\item
\item
\item
\item


\end{itemize}

\subsection*{Dealing with 1st Class Functions in JS:}
\begin{itemize}
\item Functions are mutable objects in JS. You can add properties to them after they are made.
\item First Class Function: a function that can be used the same way as variables in a program. For example, a 1st class function can:
\begin{enumerate}
\item ...be generated at run time, and not interp/compile time.
\item ...be stored in a data structure.
\item ...be passed as an argument.
\item ...returned as a value to another function.
\item ...be assigned to another reference name.

\end{enumerate}
\item \textbf{Higher Order Function:} One that accepts and returns other functions.
\item \textbf{Invoking:} is another name for calling a function.
\item Function Declaration/Statement: A block of code with a specified function name. This is stored in memory after being read.
\item Function Expression: is a block of code (anonymous - no name) that returns a value when executed.
\item When a function name is specified by itself in a statement, this references the body of code (not invoked).
\item To invoke the function, we state the name with Parens () and arguments after it. This runs the block of code and returns a value.
\item
\item
\item
\item
\item
\item

\end{itemize}

\subsection*{Glossary:}
\begin{itemize}
\item \textbf{Event Bubbling:} Refers to passing the event back up the propogation path to ancestors, \textbf{after} the target event listener(s) have activated.
\item \textbf{Event Capturing:} When an event matches an event handler that is registered to one of our DOM elements.
\item \textbf{Event Type:} What kind of action has occured on the page - "click", "blur", "load", "mouseover", etc.
\item \textbf{Bubbling:} A property an event can have. If it bubbles, it will go back up the propogation path after the target handler(s) have compelted.
\item \textbf{Capturing:} A property an event can have; this means that ancestors of the target may operate on the event, with suitable event handlers.
\item Pseudo-State: Simple subselectors that involve events, and are implemented in css. Subselector denoted by single colon (:), with the following notation:

\begin{verbatim}
  classname:click {color:#555555;...etc}
\end{verbatim}

Basic Events that can be used: Focus, Hover, Active, Visited.

\item Pseudo-Element: Allows styling of parts of elements, such as the first letter or line of a text element. subselector specified with double colon ::

\begin{verbatim}
classname::first-letter {color:red;font-size:72pt;}
\end{verbatim}
\end{itemize}

\section*{Chromium DevTools: Basic Usage:}
\begin{itemize}
\item \textbf{Tabs:}
\begin{enumerate}
\item \textbf{Elements:} For every element, we can look at attached CSS styles, Event Listeners, Properties...etc 
\item Console: This runs after all javascript compiled for page; we can run functions or define new objects in our RT env.
\item Sources: Folder structure of all sources; we can view and edit each source. Note: JS Code in static HTML document cannot be edited; put in separate file.
\item Network: Examine Gets, Post, Puts, and other requests.
\end{enumerate}
\item Debugging cccurs in the Sources Pane. There are three panels:
\begin{itemize}
\item File Navigator
\item Code Editor
\item JS Debug Pane: Watch Statements, Call Stack, Scope, Breakpoint Area, Breakpoint Selectors (various catagories and types).
\end{itemize}
\item \textbf{Watch Statements:} are global expressions that evaluate to a value, as the script runs. Depending on scope, they may or may not be defined.
\textbf{Call Stack:} Shows the functional stack upto the Break Point (BP you have set)
\textbf{Scope:} Shows the local variables involved in the block your current BP is in.
\item \textbf{Breakpoint Area:} Lists every breakpoint you have set; you can toggle on and off.
\item \textbf{Types of Break Points:} Functional, Exact line of code (click in code editor Left Side), By Event Type
\item \textbf{Very Simple Debug Method:}
\begin{enumerate}
\item ID lines of code, functions that cause problem. Set BPs.
\item Blackbox deep library calls, to not get lost in call stack (such as JQuery)
\item Examine Code Running: Look at Scope, Narrow BPs, test Hypotheses and assumptions with Watch Statements.
\item Attempt a live fix
\item Retest code again (go to 3) until problem is found.
\item Implement fix in actual code.
\end{enumerate}

\item Note: Code edited only reflects running copy; not files you are editing in an IDE/Editor. You cannot save to these files because of Sandboxing (generally)
\item
\item
\item 
\item
\item
\item
\item

\item \textbf{Things to learn later:} Code Minification, Cookie inspection and usage, Animation inspections, Tabs: Memory, Application, Security, Audits.
\end{itemize}


\begin{thebibliography}{9}

\bibitem{D3Tut}
\url{https://www.dashingd3js.com}
\bibitem{Bostok}
\url{https://bost.ocks.org/mike/join/}
\bibitem{Tim Knight}
\url{https://timmknight.github.io/2015/first-class-functions-javascript/}
\bibitem{StatementExpression}
\url{https://stackoverflow.com/questions/4728073/what-is-the-difference-between-an-expression-and-a-statement-in-python}
\bibitem{1stclassfunction1}
\url{https://hackernoon.com/javascript-and-functional-programming-an-introduction-286aa625e26d}
\bibitem{1stclassfunction2}
\url{https://hackernoon.com/javascript-and-functional-programming-pt-2-first-class-functions-4437a1aec217}
	
\end{thebibliography}


\end{document}